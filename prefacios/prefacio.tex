\chapter*{}
%\thispagestyle{empty}
%\cleardoublepage

%\thispagestyle{empty}

\input{portada/portada_2}



\cleardoublepage
\thispagestyle{empty}

\begin{center}
{\large\bfseries Nuevas propuestas de técnicas de recomendación para grupos}\\
\end{center}
\begin{center}
Juan José Sierra González\\
\end{center}

%\vspace{0.7cm}
\noindent{\textbf{Palabras clave}: sistema de recomendación, grupos, técnica, comparativa, análisis experimental}\\

\vspace{0.7cm}
\noindent{\textbf{Resumen}}\\

Los sistemas de recomendación son imprescindibles en la sociedad actual: son muy utilizados en tiendas \textit{online} y proveedores de contenido multimedia, dos grandes puntales de la actividad tecnológica de un usuario medio. En particular, la rama de la recomendación a grupos es la que más se aprovecha de la interacción entre usuarios. En este trabajo se diseñarán e implementarán cuatro nuevas propuestas de recomendación para grupos, utilizando un componente social como base sobre la que formar el algoritmo. Finalmente, a través de un análisis experimental se obtendrán resultados que ilustren su eficacia y se compararán con otras técnicas de recomendación de referencia que ayuden a valorar su potencial.
\cleardoublepage


\thispagestyle{empty}


\begin{center}
{\large\bfseries New proposals on group recommender techniques}\\
\end{center}
\begin{center}
Juan José Sierra González\\
\end{center}

%\vspace{0.7cm}
\noindent{\textbf{Keywords}: recommender system, groups, technique, comparative analysis, experimental analysis}\\

\vspace{0.7cm}
\noindent{\textbf{Abstract}}\\

Recommender systems are essential in this day and age: they are widely used in online stores and media-services providers, two main tools in a common user's technological activities. In particular, group recommender systems are the ones that make the most use of user interactions. In this work, four new group recommender proposals will be designed and implemented, using a social component as a basis for the development of the algorithm. Finally, results gathered through an experimental analysis will illustrate their efficiency, and they will be compared with other reference recommender techniques that will help valuing their potential.

\chapter*{}
\thispagestyle{empty}

\noindent\rule[-1ex]{\textwidth}{2pt}\\[4.5ex]

Yo, \textbf{Juan José Sierra González}, alumno del Máster Universitario Oficial en Ciencia de Datos e Ingeniería de Computadores de la \textbf{Escuela Técnica Superior de Ingenierías Informática y de Telecomunicación de la Universidad de Granada}, con DNI 76589592Y, autorizo la
ubicación de la siguiente copia de mi Trabajo Fin de Máster en la biblioteca del centro para que pueda ser
consultada por las personas que lo deseen.

\vspace{6cm}

\noindent Fdo: Juan José Sierra González

\vspace{2cm}

\begin{flushright}
Granada a 17 de enero de 2020.
\end{flushright}


\chapter*{}
\thispagestyle{empty}

\noindent\rule[-1ex]{\textwidth}{2pt}\\[4.5ex]

D. \textbf{Juan Manuel Fernández Luna}, Profesor del Área de Uncertainty Treatment in Artificial Intelligence del Departamento de Ciencias de la Computación e Inteligencia Artificial de la Universidad de Granada.

\vspace{0.5cm}

\textbf{Informa:}

\vspace{0.5cm}

Que el presente trabajo, titulado \textit{\textbf{Nuevas propuestas de técnicas de recomendación para grupos}},
ha sido realizado bajo su supervisión por \textbf{Juan José Sierra González}, y autoriza la defensa de dicho trabajo ante el tribunal
que corresponda.

\vspace{0.5cm}

Y para que conste, expide y firma el presente informe en Granada a 17 de enero de 2020.

\vspace{1cm}

\textbf{El tutor:}

\vspace{5cm}

\noindent \textbf{Juan Manuel Fernández Luna}

\chapter*{Agradecimientos}
\thispagestyle{empty}

\vspace{1cm}

A mi familia, en especial a mi madre y a mi hermana, por su infinita paciencia conmigo.\\

A mis amigos y compañeros informáticos, junto a los que he crecido como informático pero sobre todo como persona.\\

A mi pareja, por todo su apoyo en todos los aspectos posibles durante el transcurso del máster.\\

Y por último, pero no menos importante, a mi tutor, por su dedicación y su ayuda a la hora de realizar este trabajo.\\