\chapter{Planificación}

En este capítulo se aborda el plan de trabajo a seguir para la realización de este estudio. En primar lugar se realizará la captación de requisitos necesarios para lograr el objetivo propuesto, y después se detallará la planificación del trabajo, con una estimación en costes tanto de trabajo como de materiales así como una distribución del tiempo entre tareas.

\section{Requisitos de investigación}

Al tratarse de un trabajo de investigación, no se puede realizar adecuadamente el análisis de requisitos habitual para un proyecto general de ingeniería. En su lugar, sin embargo, se plantean unos objetivos diferentes que se deben cumplir para llevar a cabo un estudio de estas características. En el caso de este estudio en particular, los requisitos que se pueden plantear son los siguientes:

\begin{enumerate}
	\item{\textbf{Realizar una investigación inicial en torno a los sistemas de recomendación}: revisar la literatura buscando información acerca de estos sistemas, comprobar qué evolución han seguido a lo largo de los años y cómo se han llegado a desarrollar los sistemas de recomendación a grupos.}
	\item{\textbf{Profundizar en el estado del arte del sistemas de recomendación a grupos}: investigar más a fondo en estos métodos que son en los que realmente se va a centrar el estudio. Obtener los sistemas más destacados del estado del arte.}
	\item{\textbf{Seleccionar el algoritmo de referencia del estado del arte}: Entre todos los sistemas de recomendación a grupos que se hayan investigado en el paso anterior, decidir cuál es el que se ajusta mejor a los requerimientos del estudio y obtener toda la información necesaria para implementarlo.}
	\item{\textbf{Diseñar el experimento}: crear los grupos para la experimentación e implementar un algoritmo \textit{baseline} que sirva como punto de referencia. Implementar las estrategias de agregación de soluciones a la hora de experimentar y comprobar su correcto funcionamiento con dicho algoritmo.}
	\item{\textbf{Implementar el algoritmo del estado del arte}: implementar el método siguiendo los parámetros y estructura del experimento diseñado. Es necesario contar con el código de los algoritmos para el experimento, por lo que hay que implementar el del estado del arte si no se puede conseguir su código fuente.}
	\item{\textbf{Diseñar las nuevas propuestas del estudio}: para el objetivo del estudio hay que plantear nuevas propuestas que puedan competir con el estado del arte. Para ello primero hay que valorar los campos que quedan por experimentar y proponer una serie de algoritmos que se ajusten al perfil buscado.}
	\item{\textbf{Implementar las propuestas propias}: una vez decididas las propuestas que van a formar parte del experimento, se deben implementar del mismo modo que el algoritmo del estado del arte. Comprobar de nuevo que toda la funcionalidad del experimento es válida con la nueva implementación.}
	\item{\textbf{Obtención de resultados}: obtener a través de la implementación del experimento los resultados del mismo. Se pueden visualizar los datos con tablas y gráficos que sean adecuados para el estudio.}
	\item{\textbf{Estudio analítico de los resultados obtenidos}: comparar los resultados de cada método con el método de referencia \textit{baseline} y con el método del estado del arte, así como entre ellos, a fin de descubrir qué algoritmo se comporta mejor para cada caso, tanto entre grupos formados de distinta forma como con estrategias de agregación diferentes.}
	\item{\textbf{Extraer conclusiones del estudio realizado}: dar una visión analítica de la eficacia de las propuestas ideadas para el estudio, apoyándose en los resultados obtenidos con respecto a los algoritmos de referencia, y justificar cuáles de ellas son más prometedoras.}
	\item{\textbf{Trabajos futuros a realizar}: valorar en qué aspectos se puede seguir progresando en los métodos propuestos en el estudio, así como proponer nuevas tareas o retos dentro de este ámbito de investigación que no se hayan abarcado en el proceso del mismo.}
\end{enumerate}

Con los requisitos planteados, la planificación del trabajo debe ocupar los pasos que se han definido y estimar un tiempo con el que se podría solventar cada requisito. Además, debe describir todo el material e infraestructura necesarios y, junto al tiempo estimado, dar una idea del presupuesto que requiere realizar este estudio.

