\chapter{Planificación}

En este capítulo se aborda el plan de trabajo a seguir para la realización de este estudio. En primar lugar se realizará la captación de requisitos necesarios para lograr el objetivo propuesto, y después se detallará la planificación del trabajo, con una estimación en costes tanto de trabajo como de materiales así como una distribución del tiempo entre tareas.

\section{Requisitos de investigación}

Al tratarse de un trabajo de investigación, no se puede realizar adecuadamente el análisis de requisitos habitual para un proyecto general de ingeniería. En su lugar, sin embargo, se plantean unos objetivos diferentes que se deben cumplir para llevar a cabo un estudio de estas características. En el caso de este estudio en particular, los requisitos que se pueden plantear son los siguientes:

\begin{enumerate}
	\item{\textbf{Realizar una investigación inicial en torno a los sistemas de recomendación}: revisar la literatura buscando información acerca de estos sistemas, comprobar qué evolución han seguido a lo largo de los años y cómo se han llegado a desarrollar los sistemas de recomendación a grupos.}
	\item{\textbf{Profundizar en el estado del arte del sistemas de recomendación a grupos}: investigar más a fondo en estos métodos que son en los que realmente se va a centrar el estudio. Obtener los sistemas más destacados del estado del arte.}
	\item{\textbf{Seleccionar el algoritmo de referencia del estado del arte}: Entre todos los sistemas de recomendación a grupos que se hayan investigado en el paso anterior, decidir cuál es el que se ajusta mejor a los requerimientos del estudio y obtener toda la información necesaria para implementarlo.}
	\item{\textbf{Diseñar el experimento}: crear los grupos para la experimentación e implementar un algoritmo \textit{baseline} que sirva como punto de referencia. Implementar las estrategias de agregación de soluciones a la hora de experimentar y comprobar su correcto funcionamiento con dicho algoritmo.}
	\item{\textbf{Implementar el algoritmo del estado del arte}: implementar el método siguiendo los parámetros y estructura del experimento diseñado. Es necesario contar con el código de los algoritmos para el experimento, por lo que hay que implementar el del estado del arte si no se puede conseguir su código fuente.}
	\item{\textbf{Diseñar las nuevas propuestas del estudio}: para el objetivo del estudio hay que plantear nuevas propuestas que puedan competir con el estado del arte. Para ello primero hay que valorar los campos que quedan por experimentar y proponer una serie de algoritmos que se ajusten al perfil buscado.}
	\item{\textbf{Implementar las propuestas propias}: una vez decididas las propuestas que van a formar parte del experimento, se deben implementar del mismo modo que el algoritmo del estado del arte. Comprobar de nuevo que toda la funcionalidad del experimento es válida con la nueva implementación.}
	\item{\textbf{Obtención de resultados}: obtener a través de la implementación del experimento los resultados del mismo. Se pueden visualizar los datos con tablas y gráficos que sean adecuados para el estudio.}
	\item{\textbf{Estudio analítico de los resultados obtenidos}: comparar los resultados de cada método con el método de referencia \textit{baseline} y con el método del estado del arte, así como entre ellos, a fin de descubrir qué algoritmo se comporta mejor para cada caso, tanto entre grupos formados de distinta forma como con estrategias de agregación diferentes.}
	\item{\textbf{Extraer conclusiones del estudio realizado}: dar una visión analítica de la eficacia de las propuestas ideadas para el estudio, apoyándose en los resultados obtenidos con respecto a los algoritmos de referencia, y justificar cuáles de ellas son más prometedoras.}
	\item{\textbf{Trabajos futuros a realizar}: valorar en qué aspectos se puede seguir progresando en los métodos propuestos en el estudio, así como proponer nuevas tareas o retos dentro de este ámbito de investigación que no se hayan abarcado en el proceso del mismo.}
\end{enumerate}

Con los requisitos planteados, la planificación del trabajo debe ocupar los pasos que se han definido y estimar un tiempo con el que se podría solventar cada requisito. Además, debe describir todo el material e infraestructura necesarios y, junto al tiempo estimado, dar una idea del presupuesto que requiere realizar este estudio.

\section{Planificación del trabajo}

Este apartado está subdividido en estimación de costes y estimación de tiempo. En un primer lugar se valora el coste de la infraestructura utilizada, así como de todo el material que se haya utilizado y que esté disponible de forma libre y gratuita. A partir de ahí se puede realizar un presupuesto base del que partir. La estimación de tiempo debe ser realista y que permita cumplir los requisitos en un plazo competente, a fin de no necesitar aplazar los tiempos a última hora debido a una mala planificación. De ser así habría que cargar con el consecuente aumento en costes.

\subsection{Estimación de coste de materiales e infraestructura}

En primer lugar se tendrá en cuenta el ordenador con la que se ha desarrollado este estudio. Se trata de un ordenador portátil de la marca MSI, del año 2019, y que cuenta con un procesador i7-8750H capaz de llegar a 4.1GHz, así como con 16GB de memoria RAM DDR4-2666, una gráfica GeForce GTX 1060 6GB GDDR5 y dos SSD de 512GB cada uno, uno de los cuales tiene un SO Windows y el otro un SO Linux. Con este equipo se han realizado todas las tareas del experimento: documentación, revisión de la literatura y redacción de la memoria, labores de desarrollo e implementación de los algoritmos y ejecución del experimento para obtención de resultados.

Para obtener la información y documentación requerida para el trabajo ha sido necesaria una red de internet de banda ancha, disponible tanto en el lugar de desarrollo del trabajo como en las instalaciones de la Universidad de Granada cuando ha sido necesario. Gracias a los convenios que la Universidad establece con algunas bases de datos documentales como Scopus \cite{scopus-website} ha sido posible acceder a multitud de \textit{papers} y publicaciones sobre el tema a investigar. Además ha sido de mucha utilidad la página ResearchGate \cite{research-gate-website} para acceder a otros papers no encontrados en Scopus.

El desarrollo del trabajo ha sido realizado completamente sobre el sistema operativo Linux, con una distribución Pop\_OS! \cite{pop-os} basada en Ubuntu 18.10, por lo que es totalmente libre y carente de costes. Las principales herramientas de software utilizadas también han sido de código abierto por lo que no han conllevado coste adicional, utilizando como IDE principal para el desarrollo Visual Studio Code \cite{vscode-github} y escribiendo la memoria en LaTeX utilizando el software TeXstudio \cite{texstudio}, disponible para Ubuntu.

De todo este material expuesto en los párrafos anteriores, el trabajo sólo ha requerido comprar el ordenador portátil personal y la red de banda ancha disponible en el lugar de desarrollo del mismo, dado que se trata de recursos personales. El resto de la infraestructura y materiales o bien han sido productos de software libre o han sido aportados por la Universidad de Granada.

\subsection{Distribución de tiempo entre tareas}

Organizar la carga de trabajo en las distintas fases del estudio es una labor que debe plantearse previo a la realización del mismo. Con el planteamiento correctamente definido se pueden decidir algunas fechas de entrega y acotar los tiempos que se le asignan a cada tarea. Gracias a ello, el progreso del proyecto puede seguir el curso planificado, es más sencillo modularizar las tareas y así finalizar en el tiempo estimado.

En primer lugar, la labor de revisión de la literatura ocupará los primeros pasos del trabajo. El estudio debe partir de un punto en el que se conozcan correctamente los detalles de los sistemas de recomendación, con todas sus características. Por supuesto, debe haberse hecho una búsqueda de documentos que tengan dicha información, así como de aquellos \textit{papers} que contengan planteamientos de algoritmos de interés. Se seleccionan las propuestas más interesantes a priori, una vez se ha investigado lo suficiente sobre el tema y se han recabado suficientes ejemplos. La realización de esta tarea puede llevar un mes.

La documentación debe ser un proceso que se lleve de principio a fin, de manera que se vayan tomando notas y documentando adecuadamente conforme se va obteniendo la información requerida. No se puede poner una estimación a priori debido a su carácter cambiante y adaptativo en función del progreso del propio trabajo.

En cuanto al diseño del experimento, es una tarea compleja que lleva una carga de trabajo considerable. Montar todo el entorno, elegir qué estructuras de datos se van a utilizar, cómo automatizar los procesos y cómo almacenar los resultados obtenidos requiere tomar una serie de decisiones cruciales a la hora de definir el proyecto. Cometer un error en este punto supone más tiempo de corrección conforme más avance se haya realizado en el experimento, por lo que es una labor puntillosa a la que merece la pena dedicar tiempo. Al menos se plantean dos semanas o un mes de plazo para terminar toda esta estructura.

Para la implementación del sistema de recomendación del estado del arte de acuerdo al entorno experimental que se haya montado puede requerir un poco de tiempo, pero al fin y al cabo el algoritmo está definido y no debería suponer más de dos semanas seguir los pasos indicados en la publicación. En algunos casos puede haber determinadas partes que no estén bien especificadas y requiera o bien más labor de investigación o bien tomar alguna decisión propia sobre la implementación (propiamente documentada en su correspondiente apartado).

Con respecto a las nuevos métodos de recomendación para el estudio, cabe destacar que no requiere únicamente labor de implementación sino que también hay que detenerse a analizar el estado del arte y pensar en qué vertientes pueden aportarse al ámbito tratado. Aunque la mayor parte de esta tarea se pueda hacer en el apartado de revisión de la literatura, es en este punto, ya habiendo implementado el algoritmo actual, cuando es más sencillo darse cuenta de aquello que se ha echado en falta durante la investigación e implementación. Como además no se ha limitado desde el primer momento el número de nuevas propuestas que puedan surgir, se ha planteado dejar un mes para todo el proceso de análisis e implementación de dichas propuestas.

Para la ejecución del experimento y la obtención de datos se pueden utilizar un par de días o incluso una semana, a fin de recopilar en tablas los valores obtenidos y conseguir visualizar los datos de manera óptima para que reflejen los objetivos deseados. De esta forma se podrá comprobar de un vistazo sencillo el impacto de cada sistema de recomendación frente al problema planteado en cada caso.

Finalmente, extraer conclusiones de los resultados puede ser más sencillo si se conocen los algoritmos y se han trabajado correctamente, de forma que sea fácil interpretarlos y descubrir patrones en los mismos. Al ser una tarea tan diversa y poco acotada, una extracción de conclusiones puede llevar desde unos días hasta una semana.

En resumen, este trabajo de investigación puede enmarcarse en el tiempo en unos 4 meses, aunque naturalmente esa estimación puede variar en función del tiempo que se le pueda dedicar diariamente al proyecto, especialmente si se compagina con un trabajo a jornada completa. Es de importancia recordar que el tiempo invertido en el experimento también tiene un coste, igual que el de los materiales e infraestructura; en esencia se trata de otro recurso. Partiendo de la base de que este estudio es un trabajo de investigación en un campo puntero tecnológicamente, podría estimarse el coste total del trabajo como una media entre el sueldo a jornada completa de un ingeniero informático junior y el de un investigador de un departamento de universidad durante el transcurso del mismo.

Estimando para una profesión de estas características un valor aproximado de 12 euros/hora, a 4 meses trabajando (20 días al mes) y dedicándole una media de 4h diarias, se estipularía el presupuesto del proyecto en 2880 euros.