\chapter{Conclusiones y trabajo futuro}

Con la experimentación llegada a su fin, se ha de proceder a conjeturar los resultados obtenidos, no sin antes realizar un breve repaso por lo que ha supuesto este trabajo de investigación dentro del marco del Máster Universitario en Ciencia de Datos e Ingeniería de Computadores de la Universidad de Granada \cite{master-datcom}. Este Trabajo de Fin de Máster se engloba en el ámbito de los sistemas de recomendación, y aunque en el máster se imparte una asignatura sobre este tema (\textit{Sistemas de recuperación de información y de recomendación}) no pude cursarla por entrar en conflicto horario con otras asignaturas y mi horario laboral. No obstante, el tema me resultaba de gran interés y ya había cursado anteriormente en el grado la asignatura \textit{Recuperación de Información}, por lo que pedí a mi tutor, docente encargado de impartir esta asignatura en el máster, que me facilitase la información necesaria para involucrarme en el trabajo y que me sirvió de mucha ayuda a la hora de enfrentarme a este reto.

Ahora sí, una vez finalizada la experimentación, habiendo obtenido los resultados y realizado un análisis sobre los mismos, es el momento de tratar de recapitular la información que se ha descubierto mediante este estudio, haciendo énfasis en cómo se han comportado las nuevas propuestas planteadas.

En primer lugar, se ha comprobado que el \textbf{método de mayor similitud} ha sido con diferencia el más eficaz de los cuatro. Este método ha obtenido por norma general mejores resultados que las otras nuevas propuestas y, por tanto, se puede afirmar que se encuentra un escalón por encima del resto.

Dentro del grupo de las otras tres nuevas propuestas, mediante la tabla de ranking de posición en cada uno de los casos planteados en el estudio se ha estimado que el \textbf{método del más cinéfilo} ha resultado ser ligeramente mejor que los otros dos. Sin embargo, en este caso no existe tanta diferencia entre ellos como sí existía entre el método de mayor similitud y este grupo de propuestas, y se ha requerido de dicha tabla para afirmar con más confianza esta hipótesis.

Los otros dos nuevos métodos propuestos para el estudio, el del \textbf{más optimista} y el de \textbf{empatía}, pertenecen al rango más bajo, obteniendo resultados poco esperanzadores ya que no son capaces de superar con claridad al algoritmo elegido como \textit{baseline}. Si bien en casos particulares o para determinados tipos de grupos pueden ser una alternativa mejor al \textit{baseline}, en términos generales no han acabado en buena posición media, despegándose poco o incluso quedándose por debajo de este algoritmo.

Tanto el método del \textbf{más cinéfilo} como el de \textbf{mayor similitud} son capaces de superar al \textit{baseline}, como punto del que partir a la hora de resultar técnicas interesantes para un estudio. Y sin embargo, ninguno es capaz de hacer sombra al \textbf{POGRS}. Como se ha demostrado, el algoritmo del estado del arte ha resultado claro vencedor en todas y cada una de las configuraciones del experimento, con distintas estrategias de combinación y para diferentes métodos de creación de grupos, siendo indiscutiblemente el mejor algoritmo de los evaluados.

El futuro sin embargo queda abierto para estas propuestas \textbf{socioinspiradas}. Con este estudio se ha demostrado que algunas de las estrategias que más ponen en práctica los seres humanos a la hora de llegar a un acuerdo para un grupo no son las ideales, siendo por ejemplo la técnica de empatía un recurso muy utilizado por grupos de amigos y que sin embargo ha quedado relegado a un puesto muy bajo en la clasificación del estudio. Un posible trabajo futuro podría consistir en realizar un ajuste de parámetros en la actualización de pesos del método de empatía, para tratar de encontrar una configuración ideal o que al menos se adapte a un tipo de problema en concreto donde pueda destacar.

Sin embargo, dentro de este trabajo también se han planteado propuestas que se acercan a lo encontrado en el estado del arte, como el método de mayor similitud. Siendo una propuesta que parte de una lógica sencilla (identificar al usuario que más se parece al grupo únicamente basado en valoraciones anteriores), ya ofrece resultados esperanzadores considerando que se trata de la primera aproximación a este método. El rendimiento de esta técnica impulsa a pensar que pueden existir alternativas socioinspiradas no expuestas en este estudio que puedan llegar a alcanzar el estado del arte. Si bien este es solo uno de los primeros experimentos realizados con esta premisa, puede suponer de ayuda a la hora de sentar las bases de una futura investigación en este campo.

En general el trabajo realizado ha cubierto una gran parte de lo aprendido en el máster, ya que aunque no he cursado directamente la asignatura \textit{Sistemas de recuperación de información y de recomendación} he utilizado gran parte de los conocimientos de esa asignatura, que he aprendido de manera autodidacta gracias a los contenidos que me ha facilitado mi tutor. Aunque ya he realizado otros experimentos en el pasado, esta ha sido una experiencia igualmente enriquecedora ya que se trataba de un entorno totalmente nuevo para mí. La metodología de trabajo que he seguido y la toma de decisiones a lo largo de la realización del mismo han sido guiadas por lo aprendido en el máster y considero un trabajo completo, que abarca distintas fases de la investigación y que representa bastante bien una de las ramas de la ciencia de datos más aplicables hoy en día.

Valoro mucho el esfuerzo que he invertido en este Trabajo de Fin de Máster, ya que he conseguido sacarlo adelante a la vez que trabajaba a jornada completa. He puesto mucha dedicación, y aunque los resultados del estudio no hayan favorecido a las nuevas propuestas que he planteado, estoy convencido que se trata de un trabajo cuidado y bien estructurado, y que se lleva a cabo a través de un experimento correctamente planteado y ejecutado, que es lo que considero más importante a la hora de realizar un trabajo de investigación.