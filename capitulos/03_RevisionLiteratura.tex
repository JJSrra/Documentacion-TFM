\chapter{Revisión de la literatura: estado del arte}

En este capítulo se expondrá una revisión de los estudios y propuestas más innovadoras relacionadas con los \textbf{sistemas de recomendación a grupos}, y en particular, de aquellos orientados al ámbito de las películas. Se comenzará haciendo un breve inciso sobre cuándo y cómo se originaron los sistemas de recomendación, así como la particularidad de aquellos específicos de recomendación a grupos. Después se tratará su evolución hasta nuestros días, realizando un comentario analítico acerca de lo que han supuesto los sistemas de recomendación y cómo se trasladan actualmente dichos sistemas a situaciones cotidianas, con algunos claros ejemplos que todo el mundo utiliza. Entre ellos se destacarán los que abarcan la \textbf{recomendación de películas}, como tema principal del estudio. Por último, se mencionarán algunos de los sistemas de recomendación de películas a grupos más actuales, eligiendo además el que será tomado como referencia para la experimentación como representante del estado del arte de estos recomendadores.

Queda ya muy atrás el primer sistema de recomendación de la historia, que vio la luz en 1992, cuando Goldberg et al. (1992) presentaron su artículo \textit{``Using collaborative filtering to Weave an Information tapestry''} \cite{tapestry-goldberg}. En este trabajo se acuña por primera vez el término \textbf{``filtrado colaborativo''} (en inglés, \textit{collaborative filtering}), refiriéndose así a la técnica que permite filtrar elementos a un usuario basándose en lo que otros usuarios han opinado de él anteriormente. Esta técnica sienta las bases de los sistemas de recomendación, y aún a día de hoy se encuentra presente en la mayoría de ellos. La otra alternativa al filtrado colaborativo es el \textbf{filtrado basado en contenido} (en inglés, \textit{content-based filtering}). Esta técnica no tiene en cuenta las valoraciones que otros usuarios han dado a los elementos del sistema, sino que se limita a comparar los elementos por características y recomendar aquellos similares. En este experimento, sin embargo, se tratarán únicamente métodos que utilizan el filtrado colaborativo.
