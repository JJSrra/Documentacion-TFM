\chapter{Revisión de la literatura: estado del arte}

En este capítulo se expondrá una revisión de los estudios y propuestas más innovadoras relacionadas con los \textbf{sistemas de recomendación a grupos}, y en particular, de aquellos orientados al ámbito de las películas. Se comenzará haciendo un breve inciso sobre cuándo y cómo se originaron los sistemas de recomendación, así como la particularidad de aquellos específicos de recomendación a grupos. Después se tratará su evolución hasta nuestros días, realizando un comentario analítico acerca de lo que han supuesto los sistemas de recomendación y cómo se trasladan actualmente dichos sistemas a situaciones cotidianas, con algunos claros ejemplos que todo el mundo utiliza. Entre ellos se destacarán los que abarcan la \textbf{recomendación de películas}, como tema principal del estudio. Por último, se mencionarán algunos de los sistemas de recomendación de películas a grupos más actuales, eligiendo además el que será tomado como referencia para la experimentación como representante del estado del arte de estos recomendadores.

Queda ya muy atrás el primer sistema de recomendación de la historia, que vio la luz en 1992, cuando Goldberg et al. (1992) presentaron su artículo \textit{``Using collaborative filtering to Weave an Information tapestry''} \cite{tapestry-goldberg}. En este trabajo se acuña por primera vez el término \textbf{``filtrado colaborativo''} (en inglés, \textit{collaborative filtering}), refiriéndose así a la técnica que permite filtrar elementos a un usuario basándose en lo que otros usuarios han opinado de él anteriormente. Esta técnica sienta las bases de los sistemas de recomendación, y aún a día de hoy se encuentra presente en la mayoría de ellos. La otra alternativa al filtrado colaborativo es el \textbf{filtrado basado en contenido} (en inglés, \textit{content-based filtering}). Esta técnica no tiene en cuenta las valoraciones que otros usuarios han dado a los elementos del sistema, sino que se limita a comparar los elementos por características y recomendar aquellos similares. En este experimento, sin embargo, se tratarán únicamente métodos que utilizan el filtrado colaborativo.

Una vez se hubo contemplado el vasto rango de opciones que albergaban los sistemas de recomendación, a través de numerosos trabajos y estudios, se contempló la posibilidad de recomendar ya no solo a individuos, sino a grupos de personas. Se trataba de una situación a la que una sociedad podía enfrentarse en multitud de facetas, y que sin duda merecía la pena explorar. La pionera, y una de las más famosas y reconocidas propuestas sobre este tema, fue \textbf{PolyLens} \cite{polylens}. En esta publicación de 2001, O'Connor et al. plantearon \textbf{un sistema de recomendación de películas para grupos}, derivado de la base de datos de valoraciones de películas \textbf{MovieLens}, en la que varios trabajos anteriores se habían fundado. Este sistema abrió la veda de un nuevo subgrupo dentro de la ya amplia categoría de sistemas de recomendación.

Como se ha comentado anteriormente, las recomendaciones a películas han levantado mucho interés desde el primer momento. El contenido de ocio audiovisual, junto a los catálogos de tiendas \textit{online}, son los casos de mayor demanda en cuanto a sistemas de recomendación en la sociedad actual. Plataformas como \textbf{Netflix}, \textbf{Spotify} o \textbf{HBO} son ejemplos de los primeros, que utilizan estos sistemas para tratar de mantener al usuario satisfecho ofreciéndole nuevo contenido que sea similar al que hayan visto. Por su parte, \textbf{Amazon} o \textbf{AliExpress} son ejemplos de los segundos, grandes plataformas que dan cabida a todo tipo de productos y que son capaces de recomendar a sus clientes otros productos similares a los que están buscando. Gracias al sistema de recomendación pertinente, incluso consiguen ofrecer nuevos artículos que detectan que el usuario puede querer comprar, basándose en lo que anteriormente ha buscado o comprado.

Tal fue el impacto de los sistemas de recomendación que la propia plataforma Netflix lanzó en 2006 lo que se conocería como el \textbf{Netflix Prize} \cite{netflix-prize}, una competición de sistemas que, basándose únicamente en el filtrado colaborativo (no conociendo datos adicionales sobre películas ni sobre usuarios, solamente las valoraciones que estos habían dado a cada película) fueran capaces de predecir acertadamente dichas valoraciones. El objetivo principal era superar al por aquel entonces actual sistema de recomendación de Netflix, conocido como \textbf{Cinematch}, ofreciendo un premio de 1 millón de dólares a aquel sistema capaz de superarlo en un 10\%. Adicionalmente, como la competición se estimaba que duraría hasta 2011, se concedería un premio anual de 50.000 dólares al método ganador al final de cada año.

Tras varios tira y afloja entre algunos equipos, e incluso agrupaciones entre equipos para forjar mejores ideas juntos, finalmente en 2009 el equipo \textit{'BellKor's Pragmatic Chaos'} se alzó ganador y recibió el gran premio \cite{netflix-prize-winner}, superando al algoritmo anterior por 10.05\%. De esta forma tan original Netflix cambió su sistema de recomendación promoviendo una iniciativa de investigación en este campo, y por supuesto previendo que este avance en su algoritmo les serviría para incrementar la satisfacción de sus clientes, y por ende, para aumentar su valor de mercado.

Tras haber sido puestos en el panorama de investigación con casos como el Netflix Prize y el evidente aumento de su uso conforme las nuevas tecnologías se han vuelto más accesibles para la sociedad, los sistemas de recomendación han contemplado un gran abanico de variedades: algunos más enfocados a la materia que recomendar (por ejemplo, sistemas de recomendación de música o de películas), otros enfocados al tipo de usuario (sistemas de recomendación individual o a grupos) e incluso algunos cuyo propósito no es solamente obtener el mejor resultado (sistemas de recomendación explicativos que sirvan de apoyo a una decisión).

Como el objetivo de este trabajo es orientarse en recomendación a películas para grupos de usuarios, la búsqueda en la literatura de algoritmos que apoyasen el estudio se focalizó en estos dos grandes campos, analizando algunos algoritmos de la última década. La recomendación a grupos ha avanzado a muchos niveles desde la presentación de PolyLens, y ahora existen varios frentes abiertos en investigación sobre este tema.

Uno de las primeras fuentes que se recopilaron sobre los sistemas de recomendación a grupos, y que han servido a este estudio para obtener información general sobre los mismos, es el capítulo de Masthoff en el \textit{Recommender Systems Handbook} de 2011 \cite{masthoff-handbook}, donde hace un repaso a los campos de investigación abiertos de los que se hablaba en el párrafo anterior, y realiza un análisis sobre los pasos a seguir en adelante. Masthoff sienta en primer lugar los motivos por los que es necesario contar con sistemas de recomendación a grupos frente a la recomendación individual que ya se había trabajado más de una década. Aporta una detallada recopilación de las estrategias de agregación que utilizaban los sistemas de recomendación a grupos más interesantes de la época, como el ya mencionado PolyLens \cite{polylens}, el sistema MusicFX \cite{musicfx} que determina la emisora de radio que suena de fondo en un gimnasio en función de los gustos de la gente que haya ejercitándose en ese momento, o el sistema Intrigue \cite{intrigue} que ayuda a grupos de turistas a decidir qué visitar, entre otros. A continuación explica cómo se pueden aplicar los algoritmos de recomendación a grupos para usuarios individuales, dando a entender que estos algoritmos son reutilizables dependiendo del contexto en el que se necesiten. Termina planteando una serie de retos a los que los sistemas de recomendación a grupos se enfrentaban en aquel momento: aportar explicaciones a las recomendaciones propuestas, profundizar en el ``problema del inicio frío'' para grupos de usuarios o recomendar ítems en un determinado orden.

En el contexto de las recomendaciones ordenadas podrían englobarse los \textbf{rankings}, que es hacia donde han evolucionado los recomendadores individuales en las principales páginas. Es habitual no limitarse a recomendar un único ítem a un usuario, lo que propicia que se investigue en recomendar series de elementos dándoles un determinado orden (por supuesto, se quiere colocar en primer lugar aquel que se prediga que va a tener más aceptación).