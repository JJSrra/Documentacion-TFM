\chapter{Revisión de la literatura: estado del arte}

En este capítulo se realizará una revisión de los estudios y propuestas más innovadoras relacionadas con los sistemas de recomendación a grupos, y en particular, de aquellos orientados al ámbito de las películas. Se comenzará haciendo un breve repaso sobre cuándo y cómo se originaron los \textbf{sistemas de recomendación}. Después se comprobará el impacto que tuvieron los \textbf{sistemas de recomendación a grupos} a lo largo de los años y cómo evolucionaron. Por último se realizará un comentario analítico acerca de lo que han supuesto los sistemas de recomendación y cómo se trasladan actualmente dichos sistemas a situaciones cotidianas, con algunos claros ejemplos que todo el mundo utiliza. Entre ellos se destacarán los que abarcan la \textbf{recomendación de películas}, como tema principal del estudio.

Para comenzar a tratar este tema hay que remontarse a 1992, cuando Goldberg et al. (1992) presentaron en su artículo \textit{``Using collaborative filtering to Weave an Information tapestry''} el primer sistema de recomendación de la historia \cite{tapestry-goldberg}. En este trabajo se acuña por primera vez el término \textbf{``filtrado colaborativo''} (en inglés, \textit{collaborative filtering}), refiriéndose así a la técnica que permite filtrar elementos a un usuario basándose en lo que otros usuarios han opinado de él anteriormente. Esta técnica sienta las bases de los sistemas de recomendación, y aún a día de hoy se encuentra presente en la mayoría de ellos.

La otra alternativa al filtrado colaborativo es el \textbf{filtrado basado en contenido} (en inglés, \textit{content-based filtering}). Esta técnica no tiene en cuenta las valoraciones que otros usuarios han dado a los elementos del sistema, sino que se limita a comparar los elementos por características y recomendar aquellos similares.
