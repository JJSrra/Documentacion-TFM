\chapter{Nuevas propuestas de sistemas de recomendación para grupos}

En este capítulo se realizará un análisis sobre los distintos métodos ideados para competir contra el sistema de recomendación del estado del arte que se ha escogido como referencia. La idea es proponer una serie de alternativas que puedan resultar interesantes y que se espera que mejoren la eficacia de dicho método.

Como se ha mencionado en el capítulo anterior, el principal criterio que se ha seguido para escoger el método del estado del arte es, aparte de la intrínseca necesidad de tratarse de un algoritmo actual, que reproduzca en su algoritmo un comportamiento social que se asemeje a la toma de decisiones que realizan los seres humanos a la hora de enfrentarse a un problema así. En el caso del sistema de recomendación de referencia, su algoritmo se basa en decidir qué usuario es más representativo para el grupo en función de la lista de preferencias que tiene cada miembro. Para los métodos propios de este estudio se han planteado las siguientes propuestas:

\begin{itemize}
	\item \textbf{Método de empatía}
	\item \textbf{Método del más cinéfilo}
	\item \textbf{Método del más optimista}
	\item \textbf{Método de mayor similitud}
\end{itemize}

A lo largo de las próximas páginas se realizará un análisis más detallado acerca de cada uno de estos métodos, así como de dónde ha surgido la idea de cada uno y cómo puede asemejarse a un comportamiento social. Además se explicará su funcionamiento mediante pseudocódigo acompañado de una descripción del mismo en lenguaje natural.

\section{Método de empatía}

La idea para esta primera propuesta surgió desde el inicio del estudio, ya que fue un factor que no se identificó en prácticamente ningún sistema de recomendación a grupos del estado del arte. En el recomendador \textit{HappyMovie} \cite{happymovie2011} aparecía el concepto de \textit{fairness}, algo así como la ``justicia'' a la hora de realizar recomendaciones al grupo. Sin embargo, en dicha publicación no se hace referencia al método que utilizan para medir, estimar y recalcular el valor de \textit{fairness} a lo largo del tiempo. Dada la naturaleza del concepto, este valor tiene necesariamente que ser dinámico y actualizarse en función de lo que haya ocurrido en iteraciones pasadas. En este punto se decidió que una de las nuevas propuestas del estudio fuera \textbf{un sistema de recomendación a grupos capaz de ser empático} con todos los miembros del mismo.

La premisa para este algoritmo es sencilla, pero lógica: dar más importancia a las preferencias de aquellos usuarios a los que menos les hayan gustado las últimas películas recomendadas. A un usuario al que no le hayan gustado nada las últimas tres películas recomendadas se le tendrá muy en cuenta a la hora de decidir la siguiente, de la misma manera que a una persona a la que le han maravillado esas mismas películas no le importará que la siguiente le guste un poco menos.

Si se habla de un grupo de amigos que van a ver una película, este algoritmo tiene especial sentido, ya que se entiende que todos los miembros buscan el interés general del grupo además de su propio beneficio. Estar dispuesto a hacer concesiones, o ``sacrificar'' una elección óptima para que otra parte del grupo obtenga una opción mejor hace que el grupo se equilibre y puede dar lugar a una recomendación interesante desde el punto de vista algorítmico -- a través de la experimentación se dictaminará si también lo es desde el punto de vista de la eficacia.

A continuación se muestra un fragmento de pseudocódigo que representa el método de recomendación basado en empatía:
\newpage

\begin{algorithm}
	\caption{Método de empatía}
	\begin{algorithmic}[1]
		\State $valoraciones \gets predecirValoraciones()$
		\For{\textit{usuario} \textbf{in} \textit{grupo}}
		\State \textbf{Añadir} $\frac{1}{size(grupo)}$ \textbf{a} \textit{pesos}
		\EndFor
		\State $ranking \gets []$
		\While{\textit{size(películas)} \textbf{$> 1$}}
		\State $valoracionesPonderadas \gets valoraciones \times pesos$
		\State \textit{películaRecomendada} $ \gets max(valoracionesPonderadas)$
		\State $pesos \gets actualizarPesos()$
		\State \textbf{Eliminar} \textit{películaRecomendada} \textbf{de} \textit{valoraciones} \textbf{y} \textit{películas}
		\State \textbf{Añadir} \textit{películaRecomendada} \textbf{a} \textit{ranking}
		\EndWhile
		\State \textbf{Añadir última} \textit{película} \textbf{a} \textit{ranking}
		\State{\textbf{return} \textit{ranking}}
	\end{algorithmic}
\end{algorithm}

\begin{algorithm}
	\caption{Actualización de pesos para el método de empatía}
	\begin{algorithmic}[1]
		\State \textit{rango} $\gets 0.2$
		\State \textit{paso} $\gets \frac{rango}{size(grupo) - 1}$
		\State \textit{ordenSatisfacción} $\gets$ \textbf{Ordenar} \textit{valoraciones[películaRecomendada]}
		\For{\textit{posición} \textbf{in} \textit{ordenSatisfacción}}
		\State \textit{pesos[posición]} $\gets -\frac{rango}{2} + paso \times $\textit{posición}
		\EndFor
		\State \textbf{return} \textit{pesos}
	\end{algorithmic}
\end{algorithm}

En el pseudocódigo se pueden observar dos algoritmos distinguidos: el primero representa el grueso del método, con todo el proceso del sistema de recomendación y que devuelve el ranking final; el segundo es el algoritmo que actualiza los pesos para cada miembro del grupo en función de cuánto les ha gustado la película. En las siguientes líneas se entenderán qué son estos pesos y cómo afectan al criterio de selección diseñado.

En primer lugar, se analizará el algoritmo general del sistema de recomendación. Este método tiene una clara diferencia que destaca sobre el algoritmo del estado del arte, y es la presencia de \textbf{pesos} que ponderan la relevancia de cada miembro del grupo, entendiendo por relevancia lo mucho que puede influir un miembro en la toma de una decisión determinada en el transcurso de la construcción del ranking final. Es decir, estos pesos tienen que ser \textbf{dinámicos} para adaptarse a las circunstancias que se vayan aprendiendo de las iteraciones anteriores.

La inicialización de valores que se da al comienzo del algoritmo conlleva la creación de un vector de pesos que, por el momento, debe valorar a todos los usuarios por igual. Se entiende que se parte desde un punto base en el que ningún usuario está más satisfecho que otro, por lo tanto todos tienen la misma importancia a la hora de tomar una decisión sobre qué película deben escoger. Para lograr esta funcionalidad, a cada usuario de un grupo $G$ debe serle asignado un valor equivalente a $\frac{1}{|G|}$, siendo $|G|$ la cardinalidad de dicho grupo, y así conseguir que todos aporten el mismo porcentaje a la decisión final.

Para poder comenzar a recomendar cualquier película a un grupo, es necesario que se cuente con una estimación de lo que cada miembro va a valorar dicha película. Esto se consigue mediante \textbf{predicción} -- el método utilizado para ello se explicará en detalle en el apartado correspondiente a experimentación, dado que se puede sustituir fácilmente en función del estudio que se quiera realizar, y qué algoritmo se elija para predecir estas valoraciones no implica cambio alguno en cuanto a cómo se desarrolla el método.

Con la matriz de valoraciones obtenida (películas x miembros del grupo) y los pesos inicializados, se comienza un bucle del que obtener en cada iteración la película seleccionada para ser incluida en el ranking. Para ello, lo primero que hay que hacer es computar los pesos para cada usuario, dándoles a la valoración de cada uno un coeficiente de modificación que se corresponda a la importancia que han de tener en la decisión pertinente. Una vez obtenidas las valoraciones ponderadas, sumando dichas valoraciones para cada película se calcula cuál obtiene un resultado mayor, y por tanto \textbf{es la película recomendada}. El último paso de este bucle es añadir la película al ranking final (por supuesto, ordenado), eliminarla de la lista de películas y de la matriz de valoraciones, y \textbf{recalcular los pesos} asignados a cada usuario.

Al tratarse de un método un poco más complejo, se ha decidido separar este algoritmo del pseudocódigo del sistema de recomendación, y en cierto modo también realizar un paréntesis en la explicación del mismo para analizar cómo se actualizan los pesos de los usuarios en este método en particular. Como se ha indicado anteriormente, la funcionalidad principal que se persigue con este método es que los usuarios queden \textbf{satisfechos de forma empática}, es decir, se busca propiciar que los usuarios que no han visto películas que les gustan tengan más voz a la hora de escoger la siguiente. En este algoritmo se implementa una funcionalidad similar a la del algoritmo AdaBoost \cite{adaboost}, disminuyendo el peso de los miembros del grupo satisfechos y aumentando el de aquellos a los que la película les gusta menos. En este caso, los valores no pueden aumentar por encima de 1 y no pueden disminuir por debajo de 0 -- esto tiene especial sentido viéndolo desde el punto de vista del lenguaje natural como que nunca se debe ``fastidiar'' a un miembro del grupo por equilibrar la balanza, es decir, no se quiere dar una influencia negativa a tal usuario. Como mucho, si se da el caso de que dicho usuario está encantado con todas las películas, puede acabar con peso 0 significando esto que no tomará parte en la decisión de la próxima película.

El algoritmo de actualización de pesos empático funciona de la siguiente manera. En primer lugar se ha de fijar un valor para el rango de modificación de los pesos, que englobe por encima y por debajo del 0. En este caso se ha escogido el valor de rango $0.2$ tras una serie de pruebas manuales previas al experimento. Este rango provoca que la fluctuación de pesos pueda ir desde un $-0.1$ hasta un $+0.1$. Como la idea es que los miembros se distribuyan en ese espacio, se ha de calcular un \textit{paso} que especifique el escalón que ha de darse entre cada usuario.

Por ejemplo, para poder dar cabida a los 5 usuarios de un grupo G debe existir un paso equivalente a $\frac{r}{|G|-1}$, siendo $r$ el rango previamente definido y $|G|$ la cardinalidad del grupo. De esta forma, el valor del paso sería $p = 0.05$ y los usuarios se distribuirán con los valores $\{-0.1, -0.05, 0, 0.05, 0.1\}$ en función de su satisfacción con la película seleccionada.

A continuación se calcula, en función de la \textbf{valoración original} predicha para cada usuario, no de la valoración ponderada obtenida de multiplicar la predicción por los pesos, el orden de satisfacción de los usuarios con dicha película. Este orden será el que indique qué fluctuación de pesos se le otorga a los usuarios, y así el que más satisfecho haya quedado tendrá un 10\% menos de influencia en la siguiente película, así como el que menos haya quedado tendrá un 10\% más.

Por último, como se ha especificado anteriormente, es necesario limitar los valores obtenidos tras la fluctuación al rango $[0,1]$ para que la recomendación tenga sentido desde el punto de vista social en todo momento, que en última instancia es lo que se busca con este experimento; no solo obtener un resultado eficaz, sino comprobar cómo es capaz de comportarse un algoritmo que actúe de manera similar a como lo haría un ser humano.

Una vez entendido cómo funciona la actualización de pesos para el método de empatía, solo queda repetir el bucle hasta que todas las películas hayan sido clasificadas y ordenadas, y devolver el ranking correspondiente. De acuerdo al propósito del método, dicho ranking habrá tenido en cuenta mantener, a lo largo de todas las decisiones, un equilibrio que intente causar la mayor satisfacción global, pero sobre todo que logre que ningún usuario se sienta maltratado por el sistema.

Haciendo el paralelismo con la toma de decisiones humana, un ejemplo concreto de su funcionamiento sería el siguiente: el grupo de amigos habitual queda para ver películas todas las semanas, y deciden una a igualdad de condiciones. A la semana siguiente, aquellos a los que les ha gustado menos la película anterior tienen más voz a la hora de decidir una nueva película, ya que nadie quiere que sus amigos estén descontentos con la decisión tomada. En el hipotético caso de que a una persona no le hubiesen gustado nada ninguna de las dos películas anteriores, seguramente el resto del grupo la incentivaría a que escogiese una a su gusto, a fin de contentar a todos. Esto es lo que se busca con la actualización de pesos, equilibrar con el paso del tiempo las valoraciones medias de cada usuario para que no haya ninguno descolgado.

\section{Método del más cinéfilo}

El siguiente método sugiere una implementación de la idea de que es más fiable la opinión de alguien que haya visto una gran cantidad de películas. En una colección de actitudes empleadas por los seres humanos a la hora de tomar decisiones no puede faltar una de estas características, pues está claro que de forma habitual, y a veces de forma inconsciente, se da más valor a la opinión de alguien meramente por ser el más experimentado. Podría decirse que, en general, una persona que ha visto cientos de películas será capaz de aportar criterio a sus decisiones y de escoger buenas películas casi por inercia. Si bien basarse en esta premisa puede ser racional, también existe la opinión de que una persona no tiene por qué representar más al grupo simplemente por haber visto más películas. Los resultados del experimento esclarecerán esta respuesta.

En este método vuelven a aparecer los pesos, que indican qué relevancia tiene un miembro del grupo a la hora de tomar decisiones. La idea principal de este método no es más que contabilizar la cantidad de películas que un usuario ha visualizado (refiriéndose a aquellas películas del conjunto de entrenamiento, claro) y, en función del total de películas que se hayan visto en el grupo, darle un mayor o menor peso a la hora de decidir qué película seleccionar.

A continuación se expone en un fragmento de pseudocódigo el funcionamiento del método en cuestión:

\begin{algorithm}
	\caption{Método del más cinéfilo}
	\begin{algorithmic}[1]
		\State $valoraciones \gets predecirValoraciones()$
		\State $pesos \gets $ \textit{obtenerPesosCinéfilos()}
		\State $ranking \gets []$
		\State $valoraciones \gets valoraciones \times pesos$
		\While{\textit{size(películas)} \textbf{$> 1$}}
		\State \textit{películaRecomendada} $ \gets max(valoraciones)$
		\State \textbf{Eliminar} \textit{películaRecomendada} \textbf{de} \textit{valoraciones} \textbf{y} \textit{películas}
		\State \textbf{Añadir} \textit{películaRecomendada} \textbf{a} \textit{ranking}
		\EndWhile
		\State \textbf{Añadir última} \textit{película} \textbf{a} \textit{ranking}
		\State{\textbf{return} \textit{ranking}}
	\end{algorithmic}
\end{algorithm}

\begin{algorithm}
	\caption{Obtención de pesos para el método del más cinéfilo}
	\begin{algorithmic}[1]
		\For{\textit{usuario} \textbf{in} \textbf{grupo}}
		\State \textit{películasVistas[usuario]} $\gets$ \textit{obtenerPelículas(usuario)} 
		\EndFor
		\State \textit{totalPelículasVistas} $\gets$ \textit{size(películasVistas)}
		\For{\textit{usuario} \textbf{in} \textit{películasVistas}}
		\State \textit{pesos[usuario]} $\gets \frac{\textit{películasVistas[usuario]}}{\textit{totalPelículasVistas}}$
		\EndFor
		\State \textbf{return} \textit{pesos}
	\end{algorithmic}
\end{algorithm}

De nuevo se ha separado el pseudocódigo en dos partes claramente diferenciables: el cálculo de los pesos de acuerdo al método del más cinéfilo y el pseudocódigo del método en sí. En este caso sí tiene sentido explicar cómo se calculan los pesos en primer lugar, dado que esta vez \textbf{no son pesos dinámicos}. Es decir, el peso se calculará una única vez al inicio de la ejecución del algoritmo y no se volverá a recalcular durante el proceso. El cálculo de los pesos tan solo se basa en la cantidad de películas que un usuario ha visto y, como es lógico, esta cantidad no variará durante el transcurso del algoritmo.

Lo primero que hay que tener claro a la hora de decidir el grado de cinefilia de un usuario es que únicamente se puede trabajar \textbf{con las películas del conjunto de entrenamiento}. El número de películas que ha visto cada usuario se obtiene a través de dichos datos, y nunca del conjunto de prueba. Cuando se ha realizado el conteo para todos los miembros del grupo, se calcula el total de películas vistas por el grupo. El peso asignado a cada miembro corresponderá al porcentaje de películas que ha aportado él a ese total, es decir, cuántas de las películas que ha visto en total el grupo corresponden a su visionado.

Realizar el cálculo de esta manera, sin tratar con datos absolutos sino trabajando meramente con lo que hay en el grupo, permite que lo que se valore más que el número de películas total sea en realidad la diferencia existente con respecto al total de otros miembros. Así se da el valor adecuado tanto a grupos que en total han visto pocas películas como a grupos repletos de cinéfilos que hayan visto cientos de ellas. También se contempla de forma intrínseca el caso de que haya un único cinéfilo rodeado de un grupo de casi inexpertos -- en esa hipotética situación, el peso a la hora de tomar decisiones del miembro cinéfilo será sustancialmente mayor que el resto, dando a entender que su extenso rodaje puede dirigir al grupo frente a la inexperiencia de sus compañeros.

De nuevo, en este algoritmo hay que predecir las valoraciones de las películas que se van a evaluar. Como se ha indicado en el apartado anterior, este método de predicción se detallará en el capítulo de experimentación, ya que sea cual sea el escogido no afecta al desarrollo del algoritmo que se está analizando. Un detalle que cambia con respecto al algoritmo anterior es que las valoraciones pueden sobrescribirse directamente por sus equivalentes ponderadas, multiplicándolas por los pesos. Para el proceso de selección de películas de este método no volverán a hacer falta nada más que las valoraciones ponderadas, que son las que se tendrán en cuenta a la hora de decidir qué película escoger cada vez.

Con las valoraciones adecuadamente predichas y posteriormente ponderadas, se produce un bucle que terminará cuando no queden películas. En este bucle se calcula en cada iteración cuál es la película que más puntuación recibe, obteniendo dicho dato de la suma de dichas valoraciones ponderadas. Cabe recordar que estas valoraciones ya contienen la información de quiénes son los usuarios más cinéfilos y qué películas les gustan más, con lo cual se puede tratar la matriz con relativa sencillez, sin hacer mayor cálculo. Simplemente cuando se obtenga la película seleccionada, hay que eliminarla de la matriz para que no vuelva a ser escogida, así como de la lista de películas disponibles para el grupo. Esta película se incluye en el ranking de películas a recomendar.

Cuando todas las películas hayan sido incluidas por orden en el ranking, este se puede devolver como resultado final. Este ranking deberá tener en sus primeras posiciones, como característica descriptiva, una serie de películas que se consideren ``objetivamente buenas'' por las críticas, ya que la opinión de los usuarios expertos influirá mucho en la toma de decisiones. Por supuesto, esto vendrá determinado por el nivel de experiencia de los miembros del grupo, y esta característica no tiene por qué darse si no se trata con un grupo con suficiente nivel, por lo que es algo a tener en cuenta a la hora de usar este método en un problema real.

De nuevo realizando un paralelismo social de este método, un ejemplo concreto de su funcionamiento sería el siguiente: un grupo de amigos queda para ver una película, y en lugar de buscar un género concreto que pueda gustar a todos, deciden que su elección será dictaminada por aquellos miembros que hayan visto más películas, y por tanto, tengan más experiencia y más capacidad de crítica a la hora de seleccionar una. Como no quieren que sea una decisión única, se tienen en cuenta a todos los miembros del grupo pero siempre se le dará más importancia a lo que digan aquellos con una contrastada experiencia cinéfila. La ponderación inicial calculada en forma de pesos es la que representa esta importancia dentro de la toma de decisiones.

\section{Método del más optimista}

En este método se plantea una hipótesis consistente en que un usuario que valora más positivamente las películas será más propicio a disfrutarlas y a recomendar películas que sean de gusto e interés general. Si bien en algunos casos un usuario de estas características pudiera ser tachado de influenciable y poco crítico, el planteamiento que se le ha dado a este método parte de la premisa de que siempre será mejor contar con un exceso de positivismo que con una lacra del mismo. Asemejándolo al panorama social que se ha comentado en los apartados anteriores, un miembro del grupo de carácter entusiasta que valore positivamente la mayoría de las películas que ve puede transmitir al grupo una sensación de satisfacción general, mientras que una persona excesivamente crítica y al que no le convenza nada de lo que están viendo puede rápidamente minar la moral del grupo con sus valoraciones negativas.

De nuevo se le ha añadido un componente de pesos que definen cómo de optimista es un miembro del grupo con respecto al resto. De la misma forma que ocurría con los pesos del método del más cinéfilo, se calcularán en función de las valoraciones que dejen los usuarios del grupo, y no se tratará con valores absolutos a fin de adaptar el método a todos los casos posibles.

A continuación se puede apreciar un fragmento de pseudocódigo que representa el algoritmo en cuestión:
\newpage

\begin{algorithm}
	\caption{Método del más optimista}
	\begin{algorithmic}[1]
		\State $valoraciones \gets predecirValoraciones()$
		\State $pesos \gets $ \textit{obtenerPesosOptimistas()}
		\State $ranking \gets []$
		\State $valoraciones \gets valoraciones \times pesos$
		\While{\textit{size(películas)} \textbf{$> 1$}}
		\State \textit{películaRecomendada} $ \gets max(valoraciones)$
		\State \textbf{Eliminar} \textit{películaRecomendada} \textbf{de} \textit{valoraciones} \textbf{y} \textit{películas}
		\State \textbf{Añadir} \textit{películaRecomendada} \textbf{a} \textit{ranking}
		\EndWhile
		\State \textbf{Añadir última} \textit{película} \textbf{a} \textit{ranking}
		\State{\textbf{return} \textit{ranking}}
	\end{algorithmic}
\end{algorithm}

\begin{algorithm}
	\caption{Obtención de pesos para el método del más optimista}
	\begin{algorithmic}[1]
		\For{\textit{usuario} \textbf{in} \textbf{grupo}}
		\State \textit{mediaValoraciones[usuario]} $\gets$ \textit{mean(valoraciones[usuario])} 
		\EndFor
		\State \textit{totalMediaValoraciones} $\gets$ \textit{sum(mediaValoraciones)}
		\For{\textit{usuario} \textbf{in} \textit{mediaValoraciones}}
		\State \textit{pesos[usuario]} $\gets \frac{\textit{mediaValoraciones[usuario]}}{\textit{totalMediaValoraciones}}$
		\EndFor
		\State \textbf{return} \textit{pesos}
	\end{algorithmic}
\end{algorithm}

Como en casos anteriores aparece el pseudocódigo general del algoritmo separado del cálculo de los pesos para distinguir y hacer más legibles ambos fragmentos. De nuevo aquí se pueden calcular los pesos una única vez al inicio del proceso ya que no van a variar en mitad de la ejecución del algoritmo, sino basándose en las valoraciones dadas a priori.

Estas valoraciones tienen que ser extraídas también del conjunto de entrenamiento, las valoraciones del conjunto de test no tienen cabida en este momento dado que son las que se quieren predecir y se deben omitir para que no sesguen el algoritmo. En este caso lo que se busca obtener es la valoración media que los usuarios dan a las películas que ven, y a raíz de eso comprobar cómo de optimistas son respecto al cine.

Al tratar únicamente con los datos de los miembros del grupo se pueden calcular los pesos de cada usuario en relación al resto de miembros. No existe una escala predefinida que determine que si un usuario valora por encima de una cantidad \textit{X} va a ser considerado optimista y si está por debajo pesimista, sino que lo que es considerado optimista vendrá dado por las características de los usuarios que componen el grupo.

Para el cálculo de los pesos de este método se obtiene la valoración media entre todas las que ha dado cada usuario, y se suman, para obtener un total de valoraciones medias. De ese total, a cada usuario se le asigna como peso el porcentaje del total que corresponde a su valoración media, equilibrándose así de forma autónoma aquellos casos en los que haya mucha diferencia entre las valoraciones medias de los miembros del grupo, así como aquellos en los que los usuarios sean prácticamente similares, tanto si tienen valoraciones medias bajas o altas.

Con las valoraciones obtenidas mediante el algoritmo de predicción que se explicará en el capítulo de experimentación, estas son multiplicadas por los pesos obtenidos en el paso anterior, dando lugar a la matriz de valoraciones ponderadas con la que se trabajará en el resto del proceso algorítmico. Una vez ponderadas, estas valoraciones ya tienen implícita la información acerca del optimismo o pesimismo de cada uno de los usuarios, por lo que se puede trabajar directamente con ellas.

Llegado a este punto, el proceso de selección de película es similar al anterior, simplemente queda calcular la suma de las valoraciones ponderadas y así comprobar cuál es la que tiene mayor puntuación para el grupo constituido. La película elegida se incluye en el ranking de películas a recomendar al grupo, y es eliminada a su vez de la matriz y de la lista de películas disponibles, para que no pueda ser recomendada otra vez y no vuelva a aparecer en el ranking.

Cuando se hayan seleccionado todas las películas, se habrá formado un ranking que se devuelve como respuesta del sistema de recomendación. El ranking que se obtiene de este método de recomendación propicia que se seleccionen películas que le hayan gustado a aquellas personas que mejor valoran en general, por lo que se puede esperar que la tónica general de las películas que copen las primeras posiciones de la lista sean películas de esta índole, al menos en cuanto a los usuarios más optimistas del grupo. Está claro que esto no implica que vayan a gustarle forzosamente al resto de miembros, pero sí que es una estrategia de recomendación interesante desde el punto de vista social -- y que a través de la experimentación se busca averiguar si también puede operar a un buen nivel de eficacia.

El paralelismo social que se puede realizar a partir de este método puede visualizarse más claro con el siguiente ejemplo: un grupo de amigos queda para ver una película, y en lugar de basarse en el contenido de la película para elegirla deciden guiarse por las opiniones de los miembros más entusiastas del grupo, aquellos que valoran mejor las películas que ven. Esto puede venir influenciado por un gran carácter por su parte que inste al grupo a seguir su opinión, o por su entusiasmo a la hora de recomendar una película que les guste, cuando una persona que sea excesivamente crítica seguramente no animaría a los demás efusivamente a ver una película. Aún así, estas personas también tienen representación en la decisión para que nadie se sienta fuera del grupo, pero se les dará menos importancia puesto que son menos optimistas y restan moral al grupo. Con los pesos calculados en función de la valoración media dada por los usuarios a películas pasadas se puede conseguir este efecto para las tomas de decisiones del método explicado.

\section{Método de mayor similitud}

La última propuesta planteada difiere un poco de las anteriores, ya que en lugar de partir de un acuerdo entre los miembros del grupo, dándoles a cada uno de ellos una capacidad determinada de influencia en la toma de decisiones, aquí se busca identificar a \textbf{un único miembro del grupo} como el representante del mismo. En este aspecto se asemeja más al algoritmo del estado del arte, pero se enfoca de un modo distinto, como se explicará en los próximos párrafos.

Utilizar pesos resulta una forma clara de otorgar balance al grupo dándole una influencia distinta a cada miembro basándose en los datos recopilados del conjunto de entrenamiento. Para este caso no son adaptables, puesto que se quiere buscar un único perfil de usuario, y para ello hay que encontrar otra forma de decidir qué determina que un usuario represente más al grupo. En el enfoque social que se le está dando a los métodos que componen este estudio, esta propuesta resulta quizá un poco más compleja de ubicar que las anteriores, así como pasaba con el algoritmo del estado del arte escogido.

Se puede decir que en el supuesto de un grupo de amigos que van a ver una película juntos, sabiendo de sobra qué películas les gustan más y cuáles menos a cada uno de los componentes del mismo, esta propuesta trata de definir cómo el propio grupo conoce que, dadas las experiencias pasadas, las películas que más le gustan a un determinado miembro son las que más satisfecho dejan al resto. Esto puede deberse a mera casuística, o puede tener un sentido particular, como que a dicha persona le gusten las películas de un género específico o de un director en concreto.

A continuación se muestra un fragmento de pseudocódigo que ilustra el funcionamiento del método de mayor similitud:

\begin{algorithm}
	\caption{Método de mayor similitud}
	\begin{algorithmic}[1]
		\State $valoraciones \gets predecirValoraciones()$
		\State $matrizPearson \gets $ \textit{obtenerCorrelaciónPearson()}
		\State $usuarioRepresentante \gets max(mean(matrizPearson[usuario]))$
		\For{\textit{película} \textbf{in} \textit{películas}}
		\State \textbf{Añadir} \textit{valoraciones[usuario][película]} \textbf{a} \textit{ranking}
		\EndFor
		\State \textbf{Ordenar} \textit{ranking} \textbf{decreciente}
		\State \textbf{return} \textit{ranking}
	\end{algorithmic}
\end{algorithm}

A diferencia de los otros métodos, en este no aparece el pseudocódigo del cálculo de pesos aparte, ya que no tiene cabida en este método. Y además, al tratarse de un caso menos extenso en cuanto a complejidad algorítmica se ha considerado mejor incluirlo todo en un único bloque de pseudocódigo.

Lo primero como siempre es predecir las valoraciones individuales del conjunto de prueba que se quieren evaluar durante el experimento. En el capítulo dedicado a la experimentación se hará referencia con todo detalle a cómo se han predicho estas valoraciones durante todo el estudio. Por ahora, simplemente cabe recordar que estas valoraciones no se pueden obtener del conjunto de prueba sino que deben ser predichas a fin de que el algoritmo sea realmente funcional y aplicable a un problema de la vida real, y no tenga una opinión sesgada en base a dichas valoraciones.

Lo que más destaca de este método es la utilización de la \textbf{matriz de correlación de Pearson}. Si bien más adelante en la experimentación se podrá observar cómo esta matriz de Pearson ha sido necesaria en pasos previos al experimento, de cara al pseudocódigo del algoritmo se debe tratar como si se construyese en este punto. En el capítulo de experimentación se analizará detalladamente cómo se construye esta matriz, pero por ahora lo importante es destacar que representa la correlación existente entre unos y otros usuarios. Y por tanto, el funcionamiento del algoritmo consiste en obtener aquel usuario cuyo coeficiente de Pearson (es decir, su correlación) medio con respecto al resto de miembros del grupo sea mayor. De esa persona se dirá que es el que más se parece a los intereses generales del grupo, y por ello será catalogado como el \textbf{usuario representante}.

Este usuario es al que se le hará la recomendación, igual que ocurría en el método del estado del arte. Dicha recomendación, a pesar de que se hace a un único usuario y a las valoraciones predichas para él, queda como resultado del algoritmo y es el ranking global que el sistema de recomendación hace al grupo completo.

Dado que se está utilizando la fórmula de la matriz de correlación de Pearson, en este caso se está implementando una forma de reconocer en base a lo ocurrido en el pasado qué usuario es más representativo del grupo, lo cual puede lograrse en el símil social mediante la experiencia en películas pasadas. Si un grupo de amigos que ve películas habitualmente detecta que hay un miembro que se suele identificar con la impresión general del grupo (le gustan las películas que gustan al grupo en general, y análogamente con las que no gustan) no es de extrañar que se quiera dejar a dicho usuario elegir películas en el futuro. No se trata de que un usuario determinado sea capaz de predecir con toda firmeza cuánto va a valorar la película, sino que otros miembros del grupo se sientan identificados con sus elecciones pasadas. Si se da un caso así, se podría decir que dicho usuario parece representar a la mayoría del grupo, y que por tanto podría ser definido como el usuario perfil. En el capítulo de resultados se comprobará cómo de adecuada es esta propuesta en términos de eficacia.
