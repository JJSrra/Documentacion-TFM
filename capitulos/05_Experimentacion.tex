\chapter{Diseño experimental}

En este capítulo se abordarán todas las cuestiones relativas a la fase de experimentación del estudio, desde las primeras tomas de decisiones a la hora de formar grupos hasta las métricas elegidas para validar las soluciones. A lo largo de los siguientes párrafos se darán argumentos que defiendan las posturas que se han tenido que tomar y se dará una visión suficientemente detallada del experimento para que los resultados sean fácilmente interpretables.

El esquema en el que se ha basado este experimento es el que plantearon De Campos et al. en su publicación titulada \textit{``Managing uncertainty in group recommending processes''} \cite{umuai} que vio la luz en 2009. Además, más en particular, la creación de grupos ha sido bastante influenciada por lo planteado en esta publicación.

En términos generales, el experimento transcurrirá a lo largo de las siguientes fases:

\begin{enumerate}
	\item \textbf{Elección del conjunto de datos.} El primer paso necesario es seleccionar un conjunto de datos que se adecúe al objetivo del estudio; en este caso, que cuente con una buena cantidad de películas y valoraciones para poder llevar a cabo el estudio sobre sistemas de recomendación a grupos en este ámbito.
	\item \textbf{Creación de grupos.} Para llevar a cabo el experimento es necesario definir sobre qué grupos se va a trabajar y crearlos desde un primer momento, para así experimentar sobre los mismos grupos pero en diferentes métodos de recomendación. Se utilizarán dos estrategias distintas a la hora de crear grupos.
	\item \textbf{Guardado de información reutilizable de interés.} Para el funcionamiento de algunos de los métodos, o como herramienta útil en algún procesamiento interno, se guardarán algunos datos estáticos tras haberlos calculado una única vez, y no tener que emplear tiempo de ejecución innecesario cada vez que se repita el experimento. Entre estos datos se encuentran los grupos formados, pero también la matriz de correlación de Pearson o las valoraciones de cada usuario.
	\item \textbf{Filtrado de películas evaluables para cada grupo.} Al haber una gran cantidad de películas, muchas de las que ocupan el conjunto de validación pueden no haber sido vistas por ninguno de los miembros del grupo, siendo en este caso absurdo hacerlas formar parte del experimento pues no existen datos reales con los que validar la recomendación. Más adelante se especificará más concretamente qué criterio se ha seguido para escoger las películas de cada grupo.
	\item \textbf{Elección de métricas de evaluación.} Puesto que en este experimento se cuenta con soluciones en formato de ranking, existen métricas especiales capaces de interpretar mejor la bondad de una solución de este tipo. Se deben tomar también decisiones acerca de qué métricas dejar de lado por no adaptarse bien al problema en cuestión, a pesar de que puedan aparecer en otras publicaciones relacionadas.
	\item \textbf{Evaluación de los métodos del estudio.} Primero se ha de comprobar el algoritmo base, conocido como \textit{baseline}, a continuación se comprueba que la implementación del estado del arte funciona correctamente y por último se desarrollan y se evalúan las nuevas propuestas planteadas.
	\item \textbf{Extracción de resultados.} Los resultados de las distintas evaluaciones deben ser recopilados y guardados adecuadamente, a fin de poder reproducirlos más adelante en formato de gráficas o tablas que ayuden a defender las conclusiones del estudio.
\end{enumerate}

A continuación se realizará un recorrido fase por fase explicando con más precisión los distintos conceptos que se han de tener en cuenta y los factores que han propiciado decantarse por una alternativa u otra en cada caso.

\section{Elección del conjunto de datos}

Cuando se trata de evaluar un sistema de recomendación de películas, existe un conjunto de datos por excelencia que sobresale con respecto al resto: se trata de la base de datos de \textbf{MovieLens} \cite{movielens} \cite{movielens-paper}. MovieLens es un recomendador de películas utilizado en investigación por la Universidad de Minnesota, y cuenta con un contrastado bagaje ya que lleva apareciendo en publicaciones desde prácticamente los inicios de los sistemas de recomendación. Sin ir más lejos, el primer sistema de recomendación a grupos, PolyLens \cite{polylens}, se basaba en esta base de datos.

Al ser mantenido por un grupo investigador, MovieLens ofrece una amplia variedad de bases de datos con las que trabajar, más pequeñas o más grandes, y desde las más antiguas datadas de 1998 hasta las más nuevas, que se actualizan con el tiempo. De acuerdo a la información extraída de su página web, las bases de datos más antiguas son estables, pero las más nuevas son más recomendables para trabajos universitarios e investigaciones.

En base al contrastado uso de MovieLens en labores de investigación, se ha decidido escogerlo como conjunto de datos sobre el que realizar el experimento. En particular, se ha trabajado con la base de datos \textit{100K}, que cuenta con 100.836 valoraciones aplicadas sobre 9.742 películas por 610 usuarios. Estas valoraciones se obtuvieron entre marzo de 1996 y septiembre de 2018, de cuando data el conjunto en el momento de la documentación de este trabajo. El contenido descargable se compone de 4 archivos de extensión \textit{.csv}, de los cuales para este estudio solo se han requerido \textit{ratings.csv} y \textit{movies.csv}, que contienen la información acerca de las películas (nombre, año) y de las valoraciones que los usuarios han dado a qué películas.

Para el estudio se requiere que exista un conjunto de entrenamiento y uno de validación, que será con respecto al que se realizarán las recomendaciones y que deberán contrastarse con los resultados reales. Para realizar la división, se ha hecho una partición 80-20, realizando una permutación aleatoria de las 9.742 películas y asignando el 80\% al conjunto de entrenamiento (7794 películas) y el 20\% restante al conjunto de validación (1948 películas).

\section{Creación de grupos}

Para poder evaluar un sistema de recomendación a grupos se necesita agrupar a los usuarios de los que se dispone en distintas fracciones, cada una de las cuales será recomendada por el sistema a la hora de la experimentación. Para este trabajo se han determinado dos condiciones que deben cumplir todos los grupos que han de ser evaluados: la primera es que deben tener un tamaño fijo y estipulado desde el inicio; la segunda es que un usuario pertenecerá únicamente a uno de los grupos -- no puede haber un usuario que aparezca en dos grupos ni puede haber un usuario que no pertenezca a ninguno.

Pueden seguirse distintas estrategias a la hora de definir las fronteras entre miembros. De Campos et al. \cite{umuai} planteaban la existencia de formar grupos agrupando a usuarios que fuesen considerados \textit{``buddies''} o \textit{colegas}. La idea detrás de su planteamiento consistía en únicamente formar grupos con aquellos usuarios que hubieran visto al menos una película en común, pero para un número determinado previamente, igual que en este experimento.

La experimentación de esta publicación, sin embargo, se realiza \textbf{creando grupos para cada película}, ya que su método predice la valoración que un grupo le va a dar a una película en concreto, en lugar de devolver un ranking. Por este motivo no se puede aplicar a este trabajo la estrategia de creación de grupos de la misma manera exacta.

De forma empírica se ha comprobado que siguiendo este criterio para la base de datos que se ha utilizado en el trabajo aparecen pocos grupos. De tal forma, se quedan una gran cantidad de usuarios sin ser agrupados, resultando un experimento poco interesante. Sin embargo, como la idea de los grupos de colegas resulta bastante interesante, se ha decidido hacer una adaptación para este estudio. Se ha modificado la definición de ``colegas'' a ``usuarios con mayor correlación'', y se ha hecho uso del coeficiente de Pearson para obtener esta información.

El coeficiente de Pearson es un parámetro que determina cómo de correladas están dos variables, que en este caso van a ser los usuarios. Las ``características'' de estos usuarios serían las películas y las valoraciones que le han dado a cada una de ellas, pero para calcular el coeficiente entre dos personas solo se han de tener en cuenta aquellas películas que ambos hayan visto. No debe influir en la correlación entre dos usuarios que uno haya visto muchas más películas que el otro, sino cómo de parecidas son las valoraciones que han dado a películas en común que han visto. Calculando la correlación de cada usuario con respecto a los demás se puede construir la matriz de correlación de Pearson, que contiene este cálculo para cada fila y para cada columna, situándose en ambos ejes todos los usuarios del sistema. Por la propiedad simétrica de la correlación, el resultado es por supuesto una matriz simétrica, con la diagonal siendo siempre necesariamente 1 (un usuario es igual a sí mismo).

Una vez calculada esta matriz, se han generado los grupos de colegas utilizando esta correlación. Se ha fijado el tamaño del grupo a 5 usuarios, un número realista para unos amigos que se reúnen con frecuencia para ir al cine, y se ha seguido el siguiente criterio para agruparlos: se escoge al azar un usuario sin agrupar, mediante la matriz de correlación de Pearson se encuentran los 4 usuarios más correlados con él y entre todos forman un grupo y dejan de poder ser agrupados en otro. Esta es la interpretación que se le ha dado a los \textbf{grupos de colegas} para este estudio, y con este algoritmo nunca queda ningún miembro sin asignar (pues 610 es divisible entre 5).

Adicionalmente, para darle otro enfoque al experimento, se han generado también los \textbf{grupos aleatorios} seleccionando sin reemplazamiento 5 usuarios del conjunto inicial de 610 hasta que todos estén asignados a un grupo. Estas dos variantes serán las que se prueben en los sistemas de recomendación.