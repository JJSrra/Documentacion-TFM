\chapter{Introducción}

En la actualidad, la informática juega un papel fundamental en el día a día de la mayoría de personas. No solo en el ámbito profesional, donde ya lleva más años asentada, sino que cada vez es más común encontrar soluciones informáticas en problemas de la vida cotidiana. Donde antes podían requerirse varios minutos de búsqueda entre páginas de enciclopedias, ahora se resuelven consultas de todo tipo en cuestión de segundos gracias a los sistemas de acceso a los grandes volúmenes de información que se almacenan en Internet, y que comparten (y amplían) a diario millones de usuarios.

Pero la información no siempre se presenta de forma clara y concisa, y a menudo se acumulan enormes cantidades de datos aparentemente inservibles. Sin embargo, mediante el manejo de tan grandes cantidades de dichos datos se puede obtener información útil que de otra forma sería prácticamente imposible de detectar. La \textbf{ciencia de datos}, encargada de obtener toda esa información compleja, abundante y difusa y transformarla en información útil, está siendo utilizada cada vez con más asiduidad para todo tipo de finalidades. Este estudio se centrará en una de sus vertientes más en auge en los últimos años, los conocidos como \textbf{sistemas de recomendación}. Se trata de una aplicación de la ciencia de datos para recomendar elementos a usuarios, como pueden ser artículos de una tienda a través de Internet o películas de un catálogo determinado, basándose en compras o valoraciones anteriores de uno mismo o de usuarios similares.

\section{Motivación}

Los sistemas de recomendación \cite{introduction-recommender-systems} ya forman parte de la gran mayoría de plataformas web que ofrecen bienes o servicios. Mientras un usuario ojea un elemento de un catálogo en una página web es habitual que encuentre las secciones de ``Productos similares'' y ``Productos recomendados''. Mientras que la primera se basa puramente en características comunes entre los elementos, la segunda es \textbf{una recomendación personalizada} para cada usuario.

Debido a que las empresas generan beneficio directamente al realizar una buena recomendación a sus clientes, optimizar el funcionamiento de estos sistemas de recomendación resulta una tarea muy importante. Si bien los sistemas clásicos funcionan bien en términos generales, a cada empresa le interesa que su sistema se comporte de forma correcta en su ámbito, y es en estos detalles donde más interesantes son las mejoras. Y es que para una multinacional, una mejora en sus recomendaciones de un mero 1\% (que puede parecer ínfima) resulta en una inyección de ingresos muy sustancial.

Este estudio analizará uno de los ámbitos más comunes, las \textbf{recomendaciones de películas}. Como principal ejemplo, la plataforma de ocio audiovisual \textbf{Netflix} es una de las más utilizadas del momento, cuenta con un sistema de recomendación de películas y series muy elaborado, y sin embargo, solo es capaz de recomendar \textbf{para un único usuario}. No solo en el ámbito de las películas, sino en casi cualquiera, generalmente las recomendaciones se hacen de forma individual, y hay que buscar un sistema especializado si se quiere recomendar a un grupo de personas. Existen multitud de implementaciones de \textbf{sistemas de recomendación a grupos} \cite{masthoff-handbook} que tratan de optimizar la elección para satisfacer lo máximo posible a todos los miembros del grupo, o al menos lograr un equilibrio. Este trabajo se esforzará en estudiar las soluciones más actuales a este problema y cómo han mejorado lo que ofrecen las soluciones más clásicas, enmarcadas en el ámbito de las películas.

De cualquier manera, merece la pena aclarar que los sistemas de recomendación que se van a evaluar \textbf{no se tienen por qué limitar al ámbito de las películas}, y pueden trabajar con conjuntos de datos de otro tipo. Sin embargo, para este estudio se ha querido hacer hincapié en este campo en particular por resultar de especial interés en la actualidad y por tener relevancia en el desarrollo de las nuevas propuestas planteadas, como se verá más adelante.

\section{Objetivos}

Como se ha mencionado, este Trabajo de Fin de Máster pone el foco en los sistemas de recomendación de películas a grupos. Existen múltiples soluciones a este problema, que han ido mejorando con el paso de los años y el progreso logrado mediante investigación. A través de este estudio se busca entender mejor los sistemas más actuales y buscar nuevas propuestas de interés.

El primer gran objetivo es, por tanto, \textbf{estudiar las propuestas más innovadoras} del estado del arte. Mediante la revisión de la literatura y su correspondiente análisis se pueden encontrar aquellas propuestas más interesantes tanto en cuanto a exclusividad del método algorítmico como en cuanto a eficacia de la técnica en cuestión. En particular se \textbf{obtendrá una propuesta representativa} del estado del arte, que resulte equitativamente interesante en cuanto a eficacia y en cuanto a metodología.

Además, como consecuencia de este mismo estudio de las publicaciones más recientes se podrá \textbf{detectar qué ramas quedan sin explorar} en este ámbito, o en cuáles se ha puesto menos énfasis a la hora de investigar. Esto permitirá llevar a cabo el siguiente objetivo del experimento, que consiste en focalizar estos campos donde se han realizado menos publicaciones y \textbf{diseñar nuevas propuestas} relacionadas con ellos.

Aparte del diseño y el planteamiento, será necesario implementar las nuevas propuestas, así como los métodos base de referencia (un algoritmo \textit{baseline} que marque el valor mínimo a batir y el algoritmo seleccionado del estado del arte) para poder realizar un experimento empírico y obtener resultados reales. De esta forma, se podría lograr el objetivo final, consistente en \textbf{evaluar si las nuevas propuestas diseñadas son capaces de competir con el estado del arte}. Con estos resultados y a través de un análisis de los mismos se podrá obtener esta información y extraer conclusiones concretas sobre el estudio.