\chapter{Introducción}

En la actualidad, la informática juega un papel fundamental en el día a día de la mayoría de personas. No solo en el ámbito profesional, donde ya lleva más años asentada, sino que cada vez es más común encontrar soluciones informáticas en problemas de la vida cotidiana. Donde antes podían requerirse varios minutos de búsqueda entre páginas de enciclopedias, ahora se resuelven consultas de todo tipo en cuestión de segundos gracias a los grandes volúmenes de información que se almacenan en Internet, y que comparten (y amplían) a diario millones de usuarios.

Pero la información no siempre se presenta de forma clara y concisa, y a menudo se acumulan enormes cantidades de datos aparentemente inservibles. Sin embargo, mediante el manejo de tan grandes cantidades de dichos datos se puede obtener información útil que de otra forma sería prácticamente imposible de detectar. La \textbf{ciencia de datos}, encargada de obtener toda esa información compleja, abundante y difusa y transformarla en información útil, está siendo utilizada cada vez con más asiduidad para todo tipo de finalidades. Este estudio se centrará en su aplicación para \textbf{recomendar} elementos a los usuarios, o lo que más comúnmente se conoce cómo \textbf{sistemas de recomendación}.

\section{Motivación}

Los sistemas de recomendación ya forman parte de la gran mayoría de plataformas web que ofrecen bienes o servicios. Mientras un usuario ojea un elemento de un catálogo en una página web es habitual que encuentre las secciones de ``Productos similares'' y ``Productos recomendados''. Mientras que la primera se basa puramente en características comunes entre los elementos, la segunda es \textbf{una recomendación personalizada} para cada usuario.

Debido a que las empresas generan beneficio directamente de realizar una buena recomendación a sus clientes, optimizar el funcionamiento de estos sistemas de recomendación resulta una tarea muy importante. Si bien los sistemas clásicos funcionan bien en términos generales, a cada empresa le interesa que su sistema se comporte de forma correcta en su ámbito, y es en estos detalles donde más interesantes son las mejoras. Y es que para una multinacional, una mejora en sus recomendaciones de un mero 1\% (que puede parecer ínfima) resulta en una inyección de ingresos muy sustancial.

Este estudio analizará uno de los ámbitos más comunes, las \textbf{recomendaciones de películas}. Como principal ejemplo, la plataforma de ocio audiovisual \textbf{Netflix} es una de las más utilizadas del momento, cuenta con un sistema de recomendación de películas y series muy elaborado, y sin embargo, solo es capaz de recomendar \textbf{para un único usuario}. No solo en el ámbito de las películas, sino en casi cualquiera, generalmente las recomendaciones se hacen de forma individual, y hay que buscar un sistema especializado si se quiere recomendar a un grupo de personas. Existen multitud de implementaciones de \textbf{sistemas de recomendación a grupos} que tratan de optimizar la elección para satisfacer lo máximo posible a todos los miembros del grupo, o al menos lograr un equilibrio. Este trabajo se esforzará en estudiar las soluciones más actuales a este problema y cómo han mejorado lo que ofrecen las soluciones más clásicas, enmarcadas en el ámbito de las películas.

\section{Objetivos}

